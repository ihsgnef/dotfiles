\documentclass[10pt]{article}
\title{CMSC711 Midterm}
\usepackage{amsmath,amssymb,amsthm}
\usepackage{graphicx}
\usepackage{float}
\usepackage[margin=1in]{geometry}
\usepackage{fancyhdr}
\usepackage{charter}
\setlength{\parindent}{0pt}
\setlength{\parskip}{5pt plus 1pt}
\setlength{\headheight}{6pt}
\newcommand\question[2]{\vspace{.1in}\textbf{#1: #2}\vspace{.5em}\vspace{.10in}}
\renewcommand\part[1]{\vspace{.10in}\textbf{(#1)}}
\pagestyle{fancyplain}
\lhead{\textbf{\NAME}}
\chead{\textbf{Midterm}}
\rhead{CMSC711}
\begin{document}\raggedright
\newcommand\NAME{Shi Feng}

\question{1}{Search} \\
For all protocols, we assume that Bob (or adversary) is "honest but curious": he
will execute the protocol as is, but might record and explore any data that
Alice sends him. To simplify the answer, false positive documents returned to
Alice are considered as a problem of correctness of the protocol rather than extra
state stored on Alice's side. We only consider word occurrence queries for
correctness, but expand to other for the analysis of supported queries. We
assume that symmetric encryption is secure.

\part{0}
P0 has perfect recall but imperfect precision.
Alice needs to store the keys for symmetric encryption. Bob needs to store the
$m$-bit output of the Bloom Filter for each document, and $r$ hash functions.
The set size $n$ is exponential in the document length.  False positive rate is
$(\frac{m}{n})^r$ when $r=\frac{m}{n}\ln 2$. The storage over head at Bob will
be large in order to get good precision.
Bob has the hash functions, so he can find out what words appear
in each document by enumerating the set of possible words. Because Bloom Filter
ignores word order and duplicate words, this protocol only supports the word
occurrence query. It will have bad precision on phrase query and word occurrence
query with specified number of occurrences.

\part{0a}
Compared to P0, P0a has extra security guarantee that Bob cannot know the
original word occurrence in each document.
It requires extra storage of word hash function for Alice.
Other properties remain the same.

\part{1}
This protocol can correctly return all and only documents containing the search
terms. 
The only storage overhead is the hash functions on Alice's side. 
In terms of security, this is equivalent to a substitution cipher and is
susceptible to frequency analysis.
Since Bob performs string match, this protocol supports any kind of string
matching query, including word occurrence and phrase occurrence.

\part{1a}
Same as P1.

\part{2} 
This protocol can correctly return all and only documents containing search
terms. 
Alice needs to store the keys for symmetric encryption and the word hash
function; Bob needs to store the inverted index. 
Bob knows the occurrences of hashed words in each document, so he can perform
frequency analysis.
This protocol, similar to P0, only supports word occurrence queries.

\part{2a}
The chaff can act as a defense against Bob's frequency analysis.

\part{se}
The computation of the pre-encrypted word $X_i=E_{k''}(W_i)$ and the key of
stream cipher $k_i=f_{k'}(X_i)$ (replace $X_i$ with $L_i$ for scheme 4) are
independent of position, so they can be cached, or pre-computed for the whole
vocabulary.

\newpage
\question{2}{Untrusted Storage}

\newpage
\question{3}{CRL Analysis}

\newpage
\question{4}{Ephemeral Data}

\newpage
\question{5}{Trinc BlockChain}

\end{document}
